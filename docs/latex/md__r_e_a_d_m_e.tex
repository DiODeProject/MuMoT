\subsection*{Multiscale Modelling Tool }

Repository should contain following files/folders\+:
\begin{DoxyItemize}
\item {\ttfamily \hyperlink{_mu_mo_t_8py}{Mu\+Mo\+T.\+py}} (main functionality)
\item {\ttfamily process\+\_\+latex.\+py} (La\+TeX parser, imported from \href{https://github.com/augustt198/latex2sympy}{\tt latex2sympy} project and updated for Python 3)
\item {\ttfamily gen} (includes submodules used by \hyperlink{namespace_mu_mo_t}{Mu\+MoT}, important\+: there must be an empty file called {\ttfamily \+\_\+\+\_\+init\+\_\+\+\_\+.\+py} (with 2 underscores before and after init in the filename) in that folder, so Python can recognise the modules)
\item {\ttfamily \+\_\+\+\_\+mumot\+\_\+files\+\_\+\+\_\+} (also here the name of the folder is enclosed by 2 underscores before mumot\+\_\+files and 2 underscores after, this folder is a placeholder for temporary files that are generated by \hyperlink{namespace_mu_mo_t}{Mu\+MoT}, include it in the same way as done above in the master branch)
\item {\ttfamily Mu\+Mo\+Ttest.\+py} and other demo files
\end{DoxyItemize}

\section*{Dependencies\+:}

You need to install the following tools\+: Py\+D\+S\+Tool, graphviz (graph visualization) and antlr4 4.\+5.\+3 (parser generator). It is possible to use pip. Open a terminal window and type\+:


\begin{DoxyItemize}
\item {\ttfamily pip install pydstool}
\item {\ttfamily pip install graphviz}
\item {\ttfamily pip install antlr4-\/python3-\/runtime=4.\+5.\+3}
\end{DoxyItemize}

\section*{Test}

To test your installation run the {\ttfamily Mu\+Mo\+Tdemo.\+ipynb} and {\ttfamily Mu\+Mo\+Ttest.\+ipynb} notebooks.

\section*{Documentation}

Read the documentation at \href{https://diodeproject.github.io/MuMoT/}{\tt https\+://diodeproject.\+github.\+io/\+Mu\+Mo\+T/}

\section*{House rules}


\begin{DoxyItemize}
\item Update the \href{http://www.stack.nl/~dimitri/doxygen/index.html}{\tt Doxygen} documentation when making substantive changes
\begin{DoxyItemize}
\item \href{http://www.stack.nl/~dimitri/doxygen/download.html}{\tt Download} Doxygen
\item Run locally with {\ttfamily Doxyfile} configuration file from repository
\item commit the contents of {\ttfamily docs/}
\end{DoxyItemize}
\item Update {\ttfamily Mu\+Mo\+Ttest.\+ipynb} to add tests for new functionality
\item Write code using Python \href{https://www.python.org/dev/peps/pep-0008/#naming-conventions}{\tt naming conventions} 
\end{DoxyItemize}